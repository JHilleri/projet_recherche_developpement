\documentclass{article}
\usepackage[utf8]{inputenc}
\usepackage{scrextend}
\usepackage{mathtools}

\begin{document}
    \title{modélisation du problème d'ordonnancement}
    
    \author{jean hilleriteau}
    
    
    \section*{Paramètres}
        \begin{labeling}{paramètres}
        
            \item [$m$] Nombre de machines
            
            \item [$n$] Nombre de jobs. 
            Il est possible de faire des lots ne contenant qu’un seul job, 
            n désigne donc également le nombre maximal de lot.
            
            \item [$h_{i,j}^{WIP}$] Coût d’inventaire unitaire pour le job $j$ entre la machine $i$ et la machine $i+1$
            
            \item [$h_j^{FIN}$] Coût d’inventaire unitaire pour le job $j$ après la fin de la production
            
            \item [$p_{i,j}$] Durée du travail pour le job j sur la machine i
            
            \item [$C^V$] Coût par véhicule (aussi le coût par lot), représente le coût de la mobilisation d’un véhicule.
            
        \end{labeling}
    
    
    \section*{Variables}
    
        Pour représenter l’ordre des jobs lors de la production, on utilise une matrice qui indique pour chaque job s’il doit être réaliser avant un autre.
        
        Par exemple, pour la matrice suivante.

        $$
            y=
            \begin{pmatrix}
                0&1&1\\
                0&0&1\\
                0&0&0
            \end{pmatrix}
        $$
        Dans le lot 1, le job 1 est précède des jobs 2 et 3, il est donc le dernier à être effectué. Les jobs seront effectués dans l’ordre $\left\{3,2,1\right\}$.
        
        \begin{labeling}{variables}
            \item [$y_{j1,j2}$] Variable de précédences en production pour chaque lot $k$, vaut 1 si le job $j1$ est effectué avant le job $j2$. 
            \item [$PC^{local}\left(d,k\right)$] Fonction qui pour un lot $k$ calcule les pénalités de retards si la livraison débute à la date $d$. Cette fonction n’est pas encore définie.
            \item [$z_k$] Vaut 1 si le véhicule k est utilisé.
            \item [$Z_{j,k}$] Vaut 1 si le job $j$ fait partie du lot $k$
            \item [$C_{i,j}$] Date de fin de la tache $j$ pour la machine $i$
            \item [$F_k$] Date de départ du lot/véhicule $k$.
            \item [$f_j$] Date de départ du job $j$.
            \item [$IC^{WIP}$] Somme des coûts d’inventaire sur les jobs en cours
            $$IC^{WIP}=\sum_{j=1}^{n}\sum_{i=1}^{m-1}{\left(C_{i+1,j}-C_{i,j}-p_{i,j}\right)h_j^{WIP}}$$
            
            \item [$IC^{FIN}$] Somme des coûts d’inventaire sur les jobs terminé.
            $${IC}^{FIN}=\sum_{j=1}^{n}{\left(f_j-c_{m,j}\right)h_j^{FIN}}$$
            \item [$PC$] Somme des coûts des retards.
            $$PC=\sum_{k=1}^{n}{PC^{local}\left(F_k\right)}$$
            \item[$VC$] Coût des véhicules.
            $$VC=c^V\sum_{k=1}^{n}z_k$$
            
        \end{labeling}
    
    
    \section*{Objectif}
        Somme des coûts à minimiser 
        $$IC=IC^{WIP}+IC^{FIN}+PC+VC$$
        
    
    \section*{Contraintes}
    
        \begin{itemize}
            \item 
                Chaque job est soit le successeur soit le prédécesseur de chacun des autres jobs.
                \begin{align}
                    y_{j1,j2}+y_{j2,j1}=1   &&
                    \forall j1,j2\in\left\{1,\dotsc,n\right\}, j1<j2
                \end{align}
        
            \item 
                Les jobs ne peuvent pas être leurs propres successeurs.
                \begin{align}
                    y_{j,j}=0 &&
                    \forall j\in \left\{1,\cdots, n\right\}
                \end{align}
            \item
        	    Contrainte de gamme, les machines ne peuvent pas travailler sur plusieurs jobs simultanément.Assure également que les jobs sont produit dans l'ordre prescrit.
        	    \begin{align}
                    C_{i,j2}\ \geq C_{i,j1}+p_{i,j2}-My_{j1,j2} &&
                    \forall i\in\left\{1,\dotsc,m\right\}, \forall\left(j1,j2\right)\in\left\{1,\cdots,n\right\}^2
        	    \end{align}
            \item
            	Contrainte de précédence, une machine ne peut pas travailler sur un job tant que la machine précédente n’a pas terminé de travailler dessus.Assure également que les tâches des jobs sont produit dans l'ordre prescrit.
            	\begin{align}
                    C_{i,j}\geq C_{i-1,j}+p_{i,j} &&
                    \forall j\in\left\{1,\cdots,n\right\},\forall i\in\left\{2,\dotsc,m\right\}
            	\end{align}
            \item
            	Chaque job ne peut appartenir qu’à un seul lot.
            	\begin{align}
                    \sum_{k=1}^{n}Z_{j,k}=1 &&
                    \forall j\ \in\left\{1,\dotsc,n \right\}
            	\end{align}
            \item
            	Un lot ne peut être livré que si tous ces jobs sont terminés.
            	\begin{align}
                    F_k\ \geq C_{i,j}\ -\ M\left(1-Z_{j,k}\right) &&
                    \forall i \in\left\{1,\dotsc, m \right\},
                    \forall j \in\left\{1,\dotsc, n \right\},
                    \forall k \in\left\{1,\dotsc, n \right\}
            	\end{align}
            \item
            	Un lot doit exister à partir du moment où il a un job.
            	\begin{align}
                    MZ_k\geq\sum_{j=1}^{n}Z_{j,k} &&
                    \forall k\ \in\left\{1,\dotsc,n \right\}
            	\end{align}
            \item
        	Un job est envoyé en livraison en même temps que le reste du lot auquel il appartient.
                \begin{align}
                    f_j\geq F_k+M\left(1-Z_{j,k}\right) &&
                    \forall j\ \in\left\{1,\dotsc,n\right\},\ \forall k\ \in\left\{1,\dotsc,n\right\}
                \end{align}
                % correcte mais inutile
                % $$F_k \geq f_j+M\left(1-Z_{j,k}\right)$$
        \end{itemize}
    	
        
        \textbf{ Truc Truc Truc }
        
        

\end{document}
