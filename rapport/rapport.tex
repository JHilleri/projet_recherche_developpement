\documentclass{polytech/polytech}
\usepackage{scrextend}

\schooldepartment{di}
\typereport{prddi5}
\reportyear{2018-2019}
\title{Ordonnancement de production dans un problème intégrant la distribution}
\student{Jean}{Hilleriteau}{jean.hilleriteau@etu.univ-tours.fr}
\academicsupervisor[di]{Hugo}{Chevroton}{hugo.chevroton@etu.univ-tours.fr}
\resume{
    Ce projet vise à développer un algorithme pour optimiser la planification de jobs dans un problème d'atelier qui s'intègre avec un problème de livraison.
}
\abstract{
    This project is about to optimise a jobs scheduling in a flow-shop problem integrated with a salesman problem.
}
\motcle{flow-shop avec permutation}
\motcle{C++}
\motcle{recherche locale}
\motcle{resolution exacte}
\keyword{permutation flow-shop}
\keyword{C++}
\keyword{local search}
\keyword{exact resolution}

\posterblock{Objectifs}{
    L'objectif de ce projet est de comparer des méthodes pour résoudre un problème d'ordonnancement d'atelier et un problème de livraison.
}{polytech/polytech}{}
\posterblock{Mise en \oe{}uvre}{
    Le projet prend la forme d'une application codé en C++ qui implémente une méthode de résolution exacte et une recherche locale.
}{polytech/polytech}{}
\posterblock{Résultats attendus}{
    Le résultat attendu est un programme permettant de résoudre le problème d'atelier en prenant en compte son impact sur le problème de livraison.
    }{polytech/polytech}{}

\addbibresource{bibliographie}

\begin{document}

\chapter{Bilan et conclusion}

\section{Bilan semestre 9}
Durant ce premier semestre, j'ai étudié les modélisations et le code existant, j'ai également défini les algorithmes à développer.
J'ai pris du retard au début, car j'ai rencontré des difficultés à comprendre les modèles mathématiques utilisés par mon encadrant et le code existant.
Il me reste à implémenter les algorithmes et analyser leurs résultats.

\section{Bilan semestre 10}

La partie résolution exacte n'est pas terminée et la recherche locale peut être amélioré.
La recherche locale est très lente pour de grands lots, la méthode de calcul des coûts de livraisons en fonction des dates de départ des lots doit être changé.
\chapter{Bilan et conclusion}

\section{Bilan semestre 9}
Durant ce premier semestre, j'ai étudié les modélisations et le code existant, j'ai également défini les algorithmes à développer.
J'ai pris du retard au début, car j'ai rencontré des difficultés à comprendre les modèles mathématiques utilisés par mon encadrant et le code existant.
Il me reste à implémenter les algorithmes et analyser leurs résultats.

\section{Bilan semestre 10}

La partie résolution exacte n'est pas terminée et la recherche locale peut être amélioré.
La recherche locale est très lente pour de grands lots, la méthode de calcul des coûts de livraisons en fonction des dates de départ des lots doit être changé.
\chapter{Bilan et conclusion}

\section{Bilan semestre 9}
Durant ce premier semestre, j'ai étudié les modélisations et le code existant, j'ai également défini les algorithmes à développer.
J'ai pris du retard au début, car j'ai rencontré des difficultés à comprendre les modèles mathématiques utilisés par mon encadrant et le code existant.
Il me reste à implémenter les algorithmes et analyser leurs résultats.

\section{Bilan semestre 10}

La partie résolution exacte n'est pas terminée et la recherche locale peut être amélioré.
La recherche locale est très lente pour de grands lots, la méthode de calcul des coûts de livraisons en fonction des dates de départ des lots doit être changé.
\chapter{Bilan et conclusion}

\section{Bilan semestre 9}
Durant ce premier semestre, j'ai étudié les modélisations et le code existant, j'ai également défini les algorithmes à développer.
J'ai pris du retard au début, car j'ai rencontré des difficultés à comprendre les modèles mathématiques utilisés par mon encadrant et le code existant.
Il me reste à implémenter les algorithmes et analyser leurs résultats.

\section{Bilan semestre 10}

La partie résolution exacte n'est pas terminée et la recherche locale peut être amélioré.
La recherche locale est très lente pour de grands lots, la méthode de calcul des coûts de livraisons en fonction des dates de départ des lots doit être changé.

% a faire pour le s10
% \chapter{Mise en œuvre}

\chapter{Bilan et conclusion}

\section{Bilan semestre 9}
Durant ce premier semestre, j'ai étudié les modélisations et le code existant, j'ai également défini les algorithmes à développer.
J'ai pris du retard au début, car j'ai rencontré des difficultés à comprendre les modèles mathématiques utilisés par mon encadrant et le code existant.
Il me reste à implémenter les algorithmes et analyser leurs résultats.

\section{Bilan semestre 10}

La partie résolution exacte n'est pas terminée et la recherche locale peut être amélioré.
La recherche locale est très lente pour de grands lots, la méthode de calcul des coûts de livraisons en fonction des dates de départ des lots doit être changé.

\chapter{Bibliographie}

\appendix

\chapter{Bilan et conclusion}

\section{Bilan semestre 9}
Durant ce premier semestre, j'ai étudié les modélisations et le code existant, j'ai également défini les algorithmes à développer.
J'ai pris du retard au début, car j'ai rencontré des difficultés à comprendre les modèles mathématiques utilisés par mon encadrant et le code existant.
Il me reste à implémenter les algorithmes et analyser leurs résultats.

\section{Bilan semestre 10}

La partie résolution exacte n'est pas terminée et la recherche locale peut être amélioré.
La recherche locale est très lente pour de grands lots, la méthode de calcul des coûts de livraisons en fonction des dates de départ des lots doit être changé.
\chapter{Bilan et conclusion}

\section{Bilan semestre 9}
Durant ce premier semestre, j'ai étudié les modélisations et le code existant, j'ai également défini les algorithmes à développer.
J'ai pris du retard au début, car j'ai rencontré des difficultés à comprendre les modèles mathématiques utilisés par mon encadrant et le code existant.
Il me reste à implémenter les algorithmes et analyser leurs résultats.

\section{Bilan semestre 10}

La partie résolution exacte n'est pas terminée et la recherche locale peut être amélioré.
La recherche locale est très lente pour de grands lots, la méthode de calcul des coûts de livraisons en fonction des dates de départ des lots doit être changé.
\chapter{Bilan et conclusion}

\section{Bilan semestre 9}
Durant ce premier semestre, j'ai étudié les modélisations et le code existant, j'ai également défini les algorithmes à développer.
J'ai pris du retard au début, car j'ai rencontré des difficultés à comprendre les modèles mathématiques utilisés par mon encadrant et le code existant.
Il me reste à implémenter les algorithmes et analyser leurs résultats.

\section{Bilan semestre 10}

La partie résolution exacte n'est pas terminée et la recherche locale peut être amélioré.
La recherche locale est très lente pour de grands lots, la méthode de calcul des coûts de livraisons en fonction des dates de départ des lots doit être changé.
\chapter{Bilan et conclusion}

\section{Bilan semestre 9}
Durant ce premier semestre, j'ai étudié les modélisations et le code existant, j'ai également défini les algorithmes à développer.
J'ai pris du retard au début, car j'ai rencontré des difficultés à comprendre les modèles mathématiques utilisés par mon encadrant et le code existant.
Il me reste à implémenter les algorithmes et analyser leurs résultats.

\section{Bilan semestre 10}

La partie résolution exacte n'est pas terminée et la recherche locale peut être amélioré.
La recherche locale est très lente pour de grands lots, la méthode de calcul des coûts de livraisons en fonction des dates de départ des lots doit être changé.

\end{document}
