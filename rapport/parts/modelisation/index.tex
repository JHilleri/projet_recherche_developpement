\chapter{modélisation du problème global}
\label{appendix:modelisation_global}

\section{Paramètres}
\begin{labeling}{paramètres}

	\item [$m$]
	Nombre de machines
	\item [$n$]
	Nombre de jobs.
	Il est possible de faire des lots ne contenant qu’un seul job,
	n désigne donc également le nombre maximal de lot.
	\item [$h_{i,j}^{WIP}$]
	Coût d’inventaire unitaire pour le job $j$ entre la machine $i$ et la machine $i+1$
	\item [$h_j^{FIN}$]
	Coût d’inventaire unitaire pour le job $j$ après la fin de la production
	\item [$p_{i,j}$]
	Durée du travail pour le job j sur la machine i
	\item [$C^V$]
	Coût par véhicule (aussi le coût par lot), représente le coût de la mobilisation d’un véhicule.
	% paramètres pour le problème de routing
	\item [$\pi_j$]
	Cout du retard de livraison par unité de temps pour le job j
	\item [$d_j$]
	Date de livraison demandée pour le job j
	\item [$t_{j_1,j_2}$]
	Duré du trajet entre les lieux de livraison des jobs $j_1$ et $j_2$
	\item [$c_{j_1,j_2}$]
	Cout du trajet entre les lieux de livraison des jobs $j_1$ et $j_2$
\end{labeling}


\section{Variables}

Pour représenter l’ordre des jobs lors de la production, on utilise une matrice qui indique pour chaque job s’il doit être réalisé avant un autre.

Par exemple, pour la matrice suivante.

$$
	y=
	\begin{pmatrix}
		0 & 1 & 1 \\
		0 & 0 & 1 \\
		0 & 0 & 0
	\end{pmatrix}
$$
Dans le lot 1, le job 1 est précède des jobs 2 et 3, il est donc le dernier à être effectué. Les jobs seront effectués dans l’ordre $\left\{3,2,1\right\}$.

Pour représenter l'ordre de distribution des jobs, on utilise pour chaque lot, une matrice qui indique quel est le prédécesseur direct de chaque job.

Par exemple, pour deux lots $\left\{j1,j2\right\}$ et $\left\{ j3, j4\right\}$ :

$$
	\begin{pmatrix}
		0 & 1 & 0 & 0 \\
		0 & 0 & 0 & 0 \\
		0 & 0 & 0 & 0 \\
		0 & 0 & 0 & 0
	\end{pmatrix}
	\text{et}
	\begin{pmatrix}
		0 & 0 & 0 & 0 \\
		0 & 0 & 0 & 0 \\
		0 & 0 & 0 & 1 \\
		0 & 0 & 0 & 0
	\end{pmatrix}
$$
Ce qui signifie dans le lot 1, $j1$ est livré avant $j2$ et dans le lot 2, $j3$ est livré avant $j4$.

\begin{labeling}{variables}
	\item [$y_{j1,j2}$] Variable de précédences en production, vaut 1 si le job $j1$ est effectué avant le job $j2$.
	% \item [$PC^{local}\left(d,k\right)$] Fonction qui pour un lot $k$ calcule les pénalités de retards si la livraison débute à la date $d$. Cette fonction n’est pas encore définie.
	\item [$z_k$] Vaut 1 si le véhicule k est utilisé.
	\item [$Z_{j,k}$] Vaut 1 si le job $j$ fait partie du lot $k$
	\item [$C_{i,j}$] Date de fin de la tâche $j$ pour la machine $i$
	\item [$F_k$] Date de départ du lot/véhicule $k$.
	\item [$f_j$] Date de départ du job $j$.
	\item [$IC^{WIP}$] Somme des coûts d’inventaire sur les jobs en cours
	$$IC^{WIP}=\sum_{j=1}^{n}\sum_{i=1}^{m-1}{\left(C_{i+1,j}-C_{i,j}-p_{i+1,j}\right)h_j^{WIP}}$$

	\item [$IC^{FIN}$] Somme des coûts d’inventaire sur les jobs terminés.
	$${IC}^{FIN}=\sum_{j=1}^{n}{\left(f_j-C_{m,j}\right)h_j^{FIN}}$$
	% \item [$PC$] Somme des coûts des retards.
	% $$PC=\sum_{k=1}^{n}{PC^{local}\left(F_k, k\right)}$$
	\item[$VC$] Coût des véhicules.
	$$VC=c^V\sum_{k=1}^{n}z_k$$

	% variables pour le routing
	\item [$x_{j1,j2,k}$] Variable de précédence pour la livraison, vaut 1 si $j1$ est livré juste avant $j2$ et que $j1$ et $j2$ font partie du lot k.
	\item [$D_j$] Date de livraison réalisé pour le job j
	\item [$T_j$] Retard de la livraison du job j
	% $$T_j\geq D_j-d_j$$
	\item [$RC$] Cout des trajets de la livraison, sans compter le retour au dépôt après la tournée
	$$RC=\sum_{k=1}^{n}{\sum_{j_1=0}^{n}\sum_{j_2=1}^{n+1}x_{j_1,j_2,k}c_{j_1,j_2}}$$
	\item [$PC$]	Cout total des pénalités de retard
	$$PC=\sum_{j=1}^{n} T_j$$


\end{labeling}


\section{Objectif}
Somme des coûts à minimiser
$$IC=IC^{WIP}+IC^{FIN}+PC+VC+RC$$


\section{Contraintes}

\begin{itemize}
	\item
	      Chaque job est soit le successeur soit le prédécesseur de chacun des autres jobs.
	      \begin{align}
		      y_{j1,j2}+y_{j2,j1}=1 &  &
		      \forall j1,j2\in\left\{1,\dotsc,n\right\}, j1<j2
	      \end{align}
	\item
	      Les jobs ne peuvent pas être leurs propres successeurs.
	      \begin{align}
		      y_{j,j}=0 &  &
		      \forall j\in \left\{1,\cdots, n\right\}
	      \end{align}
	\item
	      Contrainte de gamme, les machines ne peuvent pas travailler sur plusieurs jobs simultanément.
	      Cette contrainte assure également que les jobs soient produits dans l'ordre prescrit.
	      \begin{align}
		      C_{i,j2} \geq C_{i,j1}+p_{i,j2}-M y_{j1,j2} &  &
		      \forall i\in\left\{1,\dotsc,m\right\}, \forall\left(j1,j2\right)\in\left\{1,\cdots,n\right\}^2
	      \end{align}
	\item
	      Contrainte de précédence, une machine ne peut pas travailler sur un job tant que la machine précédente n’a pas terminé de travailler dessus.
	      Cette contrainte assure également que les tâches des jobs soient produit dans l'ordre prescrit.
	      \begin{align}
		      C_{i,j}\geq C_{i-1,j}+p_{i,j} &  &
		      \forall j\in\left\{1,\cdots,n\right\},\forall i\in\left\{2,\dotsc,m\right\}
	      \end{align}
	\item
	      Chaque job appartient à exactement un lot.
	      \begin{align}
		      \sum_{k=1}^{n}Z_{j,k}=1 &  &
		      \forall j \in\left\{1,\dotsc,n \right\}
		\end{align}
	\item 
		Un job est terminé quand toutes ces tâches sont terminées.
		\begin{align}
			f_j \geq C_{m, j} & & \forall j \in \left\{0,\dotsc,n\right\}
		\end{align}
	\item
	      Un lot ne peut être livré que si tous ces jobs sont terminés.
	      \begin{align}
		      F_k \geq f_{j} - M\left(1-Z_{j,k}\right) &  & \forall j \in & \left\{1,\dotsc, n \right\}, \notag \\
		                                                 &  & \forall k \in & \left\{1,\dotsc, n \right\}
	      \end{align}
	\item
	      Un lot doit exister à partir du moment où il a un job.
	      \begin{align}
		      % ici on utilise n au lieux de M car la somme ne peut pas dépasser n
		      % l'utilisation de n permet d'éviter d'utiliser M qui est plus lourd a utiliser
		      n z_k\geq\sum_{j=1}^{n}Z_{j,k} &  &
		      \forall k \in\left\{1,\dotsc,n \right\}
	      \end{align}

	      % contraintes du problème de routing
	\item
	      Si un job fait partie d’un lot, il a un successeur direct dans ce lot.
	      \begin{align}
		      \sum_{j2=1}^{n+1} x_{j1,j2,k} = Z_{j1,k} &  &
		      \forall j1 \in\{ 1, \ldots, n\}, \forall k \in\{1, \ldots, n\}
	      \end{align}
	\item
	      Si un job fait partie d’un lot, pour la livraison, il a un prédécesseur direct dans ce lot.
	      \begin{align}
		      \sum_{j1=0}^{n} x_{j1,j2,k} = Z_{j2,k} &  &
		      \forall j2 \in\{ 1, \ldots, n\}, \forall k \in\{1, \ldots, n\}
	      \end{align}
	\item
	      Si un job est livré avant un autre, il arrive plus tôt chez le client.
	      \begin{align}
		      D_{j2}\geq D_{j1} + t_{j1,j2} - M \left(1-x_{j1,j2,k} \right)
			& &
			\forall (j1, j2) \in \{ 1, \ldots, n \}^2
	      \end{align}
	\item
	      Les retards de livraisons ne peuvent pas être négatifs,
	      les livraisons en avance n’ont pas d’impacts.
	      \begin{align}
		      T_j & \geq 0         &  & \forall j \in\{1, \ldots, n\} \\
		      T_j & \geq T_j \pi_j &  & \forall j \in\{1, \ldots, n\}
	      \end{align}
	\item
	      Un job ne peut pas être livré avant son départ en livraison.
		% Le trajet directe d'un lieux a vers un lieux b est forcément plus court que le trajet a vers b en passant par c.
	      \begin{align}
		      D_j \geq f_j + t_{0, j} &  & \forall j \in\{1, \ldots, n\}
	      \end{align}
\end{itemize}
