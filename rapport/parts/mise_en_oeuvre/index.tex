\chapter{mise en oeuvre}
\label{appendix:mise_en_oeuvre}

\section{Outils}

% Préciser ici Les outils et librairies utilisées (rapidement : les détails sur les versions exactes chose à savoir étant en annexe)

Le cœur du projet est écrit en C++ et un outils de visualisation des solution à été réalisé en javascript.
Les outils suivant ont été utilisé :
\subsection{Git}
J'ai utiliser git pour versionner le code du projet.
Git est un outils de gestion de version populaire, il permet de conserver un historiques des modifications apporté au projet.
\subsection{Github}
GitHub est une plateforme d'hébergement de dépôt git qui propose également des outils pour gérer et suivre les taches à réaliser et les demandes des utilisateurs.
\subsection{Visual studio}
Visual studio est un environnement de développement qui intègre les outils nécessaire au développement d'application en C++ et C\#.
\subsection{Tex Live}
Tex Live est une distribution latex qui est utilisé dans ce projet pour générer les rapports.
\subsection{Visual studio code}
Visual Studio Code est un éditeur de code modulaire qui à été utilisé dans ce projet pour la rédaction des document en Latex, pour le développement de l'outils de visualisation de solution et pour lire les données brutes des fichiers solutions.
\subsection{Doxygen}
Doxygen est l'outil qui à été utilisé pour générer la documentation du code du projet à partir de commentaires présents dans le code.
\section{Implémentations}

Le projet à été séparé en 4 sous-parties :
- implémentation de la recherche locale
- implémentation de la méthode exacte avec CPLEX
- bibliothèque de fonctionnalités communes
- tests avec le Framework GTest

Avec le projet, un petit outils pour visualiser les données des solutions sous forme de diagramme de Gantt à été créé en HTML/javascript.

\subsection{limites}

Pour respecter les demandes du client, le projet prend la forme d'une solution Visual Studio,
cela facilite sont utilisation sous Windows mais rend impossible l'utilisation du projet sous d'autres systèmes d'exploitations.
La partie de la méthode exacte nécessite la bibliothèque CPLEX lors de la compilation.

\subsection{risques}

\subsection{choix techniques}
Les solutions sont exporté en suivant le forma JSON, 
ce qui permet de les lire facilement, 
aussi bien pour des utilisateur que pour des programmes.

\subsection{devation}
La méthode exacte donne des solutions invalides.

\section{Qualité/performance}

\subsection{analyse des résultats}

des éléments sur l'implémentation (limites, risques, choix techniques, déviation/analyse…) 
des éléments sur la qualité/l'évaluation des performance (stratégie et analyse des résultats)

