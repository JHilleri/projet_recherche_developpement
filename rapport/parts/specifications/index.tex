\chapter{Spécifications fonctionnelles}


\section{Lecture des paramètre}
La première étape du programme est de lire des paramètres donnés au lancement du programme.
Le programme attend les paramètres suivant :
\begin{itemize}
    \item le chemin vers le fichier qui décrit l'instance à résoudre.
    \item l'algorithme à utiliser.
    \item Pour la recherche locale, le temps accordé à l'algorithme.
\end{itemize}

\section{Lecture de l'instance}
Cette fonction permet de lire le fichier de l'instance à résoudre, elle a déjà été réalisée dans le programme existant fournit par mon encadrant.

\section{Résolution de l'instance}
Cette fonction est la partie principale du projet.
Elle permet de résoudre l'instance avec l'algorithme choisie par l'utilisateur.
Les algorithmes à implémenter sont :
\begin{itemize}
    \item La méthode exacte présentée dans la partie \autoref{section:analyse:methode_exacte}.
    \item La recherche locale présenté dans la partie \autoref{section:analyse:methode_heuristique}.
\end{itemize}
La résolution prend une instance en paramètre.
Les données retournées sont :
\begin{itemize}
    \item la solution.
    \item le score de la solution.
    \item le temps écoulé.
    \item la mémoire utilisée.
\end{itemize}

\section{Ecriture des résultats}
Cette fonction à pour rôle d'écrire les résultats de l'algorithme dans la console.
La fonction prend en paramètre les données fournies par la résolution de l'instance et les écrit dans la sortie standard.
Les données sont :
\begin{itemize}
    \item le score de la solution.
    \item la solution.
    \item la durée de la résolution.
    \item la mémoire utilisée pour la résolution.
\end{itemize}
