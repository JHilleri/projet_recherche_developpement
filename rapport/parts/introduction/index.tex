\chapter{Introduction}

\section{Acteurs, enjeux et contexte}
Ce projet reprend le travail de mon encadrant, Hugo Chevroton, 
    qui travaille sur l'intégration de problèmes de routing dans des problèmes d'ateliers.
    L'objectif de cette approche est de fournir de meilleures solutions.
Dans ce projet, je travaille sur une sous-partie de ce problème.

\subsection{Présentation du sujet général}
Dans une entreprise qui réalise des commandes et les fait livrer chez ces clients,
    il faut déterminer dans quel ordre réaliser les commandes et planifier les itinéraires des livreurs.
Ces deux problèmes sont généralement résolus séparément.
Le travail de mon encadrant porte sur l'intégration de ces problèmes en un seul,
     de façon a trouver de meilleures solutions.

Les commandes sont réalisées dans un atelier, 
    elles passent toutes successivement par les mêmes étapes de production, dans le même ordre.
Dans le cadre de ce projet, on nomme les commandes jobs et les étapes de productions des machines.
Le travail que doit réaliser une machine sur un job est appelé une tâche.

On considère qu'une machine ne peut travailler que sur un job à la fois 
    et qu'un job ne peut être que sur une machine a la fois.
La durée des tâches peut varier selon les machines et selon les jobs.

Lorsqu'un job est en attente entre deux machines, il engendre des coûts d'inventaires.
On distingues les couts d'inventaire des jobs pendant la production 
    et les couts d'inventaire des jobs terminé en attente de livraison.

Les jobs sont répartis en lots, une fois que tous les jobs d'un lot sont terminés, ils partent en livraison.
Chaque lot est livré par un même véhicule en une tournée.

Pour résoudre ce problèmes, il y a quatre degrés de liberté que l'on peut utiliser :
\begin{itemize}
    \item Ordre de réalisation des jobs.
    \item Mise en attente de la production pour réduire les couts d'inventaires.
    \item Constitution des lots.
    \item Ordre de distributions des jobs pour chaque lot.
\end{itemize}

La modélisation du problème général est présentée dans la partie \autoref{appendix:modelisation_global}.

\subsection{Acteurs}
Ce projet est un sujet de recherche, mon encadrant est le seul client.

\section{Objectifs}
Mon encadrant a déjà résolue le sous-problème où l'ordre des jobs et la constitution des lots sont fixées.
L'objectif de ce projet est d'ajouter la possibilité de changer l'ordre des jobs.

Il faut donc résoudre simultanément les problèmes de routing et d'atelier pour des problèmes où les lots sont fixes à l'avance.

Dans un premier temps je vais utiliser un solveur.
Ensuite, je vais réaliser un programme en C++ pour résoudre le problème avec des heuristiques.

\section{Hypothèses}
Dans ce projet, on considère que l'atelier de production est constitué de plusieurs machines 
    et que tous les jobs doivent passer successivement sur chaque machine dans le même ordre.

Les durées des tâches sont connues a l'avance.
Les durées et les coûts des trajets entre les lieux de livraisons sont connues à l'avance 
    et fixes (le trajet ne coûte pas plus si le véhicule est remplie que s'il est vide).

Dans un premier temps, je vais utiliser un solveur pour résoudre le problème, 
    s'il ne permet pas de le résoudre suffisamment rapidement, 
    je vais faire un programme en C++ qui intégrera des heuristiques.

\section{Bases méthodologiques}
\subsection{outils de gestion de version}
L'ensemble du projet est hébergé sur Github (\url{https://github.com/JHilleri/projet_recherche_developpement}).

\subsection{gestion de projet}
La gestion de projet suit une méthode agile et s'appuie sur les outils de gestion de projet fourni par Github.

Chaque semaine, les tâches à réaliser sont définies avec mon encadrant.
Ces tâches sont ajoutées sur github dans l'onglet Issues et leurs avancements est suivit sur un tableau kanban.

Lorsque je termine une tache, j'ouvre un "pull request" pour que mon encadrant valide mon travail.
Une fois mon travail validé, j'intègre mes modifications à la branche principale du projet.

\subsection{déploiement continu}
Sur Github, seul les code sources du programme de résolution du problème et le rapport est disponible.
Les livrables sont disponibles à l'adresse \url{http://prd.jhilleri.ovh}.
Sur ce site, on trouve les livrables pour les différentes branches du projet.
J'utilise l'outils CircleCI (\url{https://circleci.com}) pour l'intégration continu 
    et le déploiement automatique des livrables.
Le rapport est généré avec l'image docker aergus/latex qui fournit une installation complète de texlive.
