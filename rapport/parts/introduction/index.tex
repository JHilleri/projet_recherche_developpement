\chapter{Introduction}

\section{Acteurs, enjeux et contexte}
Ce projet reprend le travail de mon encadrant, Hugo Chevroton, 
    qui travail sur l'intégration de problèmes de routing dans des problèmes d'ateliers.
    L'objectif de cette approche est de fournir de meilleurs solutions.
Dans ce projet, je travail sur une sous-partie de ce problème.

\subsection{Présentation du sujet générale}
Dans une entreprise qui réalise des commandes et les fait livrer chez ces clients,
    il faut déterminer dans quel ordre réaliser les commandes et planifier les itinéraires des livreurs.
Ces deux problèmes sont généralement résolus séparément.
Le travail de mon encadrant porte sur l'intégration de ces problèmes en un seul,
     de façon a trouver de meilleurs solutions.

Les commandes sont réalisées dans un atelier, 
    elles passent toutes successivement par les mêmes étapes de production, dans le même ordre.
Dans le cadre de ce projet, on nomme les commandes jobs et les étapes de productions des machines.
Le travail que doit réaliser une machine sur un job est appelé une tache.

On considère qu'une machine ne peut travailler que sur un job à la fois 
    et qu'un job ne peut être que sur une machine a la fois.
La durée des taches peut varier selon les machines et selon les jobs.

Lorsqu'un job est en attente entre deux machines, il engendre des coûts d'inventaires.
On distingues les couts d'inventaire des jobs pendant la production 
    et les couts d'inventaire des jobs terminé en attente de livraison.

Les jobs sont répartis en lots, une fois que tous les jobs d'un lot sont terminé, ils partent en livraison.
Chaque lot est livré par un même véhicule en une tourné.

Pour résoudre ce problèmes, il y a quatre degrés de liberté que l'on peut utiliser :
\begin{itemize}
    \item Ordre de réalisation des jobs.
    \item Mise en attente de la production pour réduire les couts d'inventaires.
    \item Constitution des lots.
    \item Ordre de distributions des jobs pour chaque lot.
\end{itemize}

La modélisation du problème générale est présenté dans la partie \autoref{appendix:modelisation_global}.

\subsection{Acteurs}
Ce projet est un sujet de recherche, mon encadrant est le seul client.

\section{Objectifs}
Mon encadrant a déjà résolue le problème de routing pour les cas où les autres degré de liberté sont fixé.
L'objectif de ce projet est d'ajouter la possibilité de retarder le début de la production de certains jobs 
    pour améliorer le résultat.

Il faut donc résoudre simultanément le problème de routing et celui des décalages des dates de début des jobs 
    pour des problèmes où les ordres des jobs et les lots sont fixe.

Dans un premier temps je vais utiliser un solveur.
Ensuit, je vais réaliser un programme en C++ pour résoudre le problème avec des heuristiques.

\section{Hypothèses}
Dans ce projet, on considère que l'atelier de production est constitué de plusieurs machines 
    et que touts les jobs doivent passer successivement sur chaque machine dans le même ordre.

Les durées des taches sont connues a l'avances.
Les durées et les coûts des trajets entre les lieux de livraisons sont connues à l'avance 
    et fixes (le trajet ne coûte pas plus si le véhicule est remplie que s'il est vide).


\section{Bases méthodologiques}
\subsection{outils de gestion de version}
L'ensemble du projet est stocké sur Github (\url{https://github.com/JHilleri/projet_recherche_developpement}).

\subsection{gestion de projet}
La gestion de projet suit une méthode agile et s'appuie sur les outils de gestion de projet fourni par Github.

Chaque semaine, les tâches à réaliser sont définies avec mon encadrant.
Ces taches sont ajoutées sur github dans l'onglet Issues et leurs avancements est visible sur un tableau kanban.

Lorsque je termine une tache, j'ouvre un "pull request" pour que mon encadrant valide mon travail.
Une fois mon travail validé, j'intègre mes modifications à la branche principale du projet.

\subsection{déploiement continu}
Sur Github, seul les code sources du programme de résolution du problème et le rapport est disponible.
Les livrables sont disponibles à l'adresse \url{http://prd.jhilleri.ovh}.
Sur ce site, on trouve les livrables pour les différentes branches du projet.
J'utilise l'outils CircleCI (\url{https://circleci.com}) pour l'intégration continu 
    et le déploiement automatique des livrables.
Le rapport est généré avec l'image docker aergus/latex qui fournit une installation complète de texlive.
