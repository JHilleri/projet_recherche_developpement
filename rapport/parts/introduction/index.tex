\chapter{Introduction}

\section{Acteurs, enjeux et contexte}
Ce projet reprend le travail de Hugo Chevroton, mon encadrant, le but est de résoudre une sous-partie du problème sur lequel il travail.

Dans une entreprise qui réalise des commandes et les fait livrer chez les clients, il faut déterminer dans quel ordre réaliser les commandes et planifier les itinéraires des livreurs.

Ces deux problèmes sont généralement résolus séparément.
Ce projet se situe dans le cas où l'on souhaite résoudre les deux problèmes ensemble, pour trouver de meilleures solutions.

Ce projet est encadré par Hugo Chevroton, qui remplie également le rôle du client.

Les commandes sont réalisées dans un atelier, elles passent toutes successivement par les mêmes étapes de production, dans le même ordre.
Dans le cadre de ce projet, on appel les commandes jobs et les étapes de productions des machines.
Le travail d'une machine sur un job est appelé tache.

On considère qu'une machine ne peut travailler que sur un job à la fois et qu'un job ne peut être que sur une machine a la fois.
Le temps de travail pour chaque machine varie selon les jobs.

Lorsqu'un job est en attente entre deux machines, il y a des coûts d'inventaires.

Les jobs sont répartis en lots, une fois que tous les jobs d'un lot sont terminé, ils partent en livraison.
Chaque lot est livré par un même véhicule en une tourné.

La modélisation du problème générale est disponible dans la partie \autoref{appendix:modelisation_global}.

\section{Objectifs}
Il est déjà possible de résoudre le problème de routing pour le cas où les autres degrés de liberté sont fixé.
L'objectif de ce projet est d'ajouter la possibilité de retarder le début de la production de certains jobs pour améliorer le résultat.

Il faudra donc résoudre ensemble le problème de routing et celui des décalages des dates de début des jobs pour des problèmes où les ordres des jobs et les lots sont fixe.

Dans un premier temps je vais utiliser un solveur, ensuit, je vais réaliser un programme en C++ pour résoudre le problème avec des heuristiques.

\section{Hypothèses}
Dans ce projet, on considère que l'atelier de production est constitué de plusieurs machines et que touts les jobs doivent passer successivement sur chaque machine dans le même ordre.

Les durées des taches sont connues a l'avances.
Les durées et les coûts des trajets entre les lieux de livraisons sont connues et fixes (le trajet ne coûte pas plus si le véhicule est remplie que s'il est vide).


\section{Bases méthodologiques}
\subsection{outils de gestion de version}
L'ensemble du projet est stocké sur Github (\url{https://github.com/JHilleri/projet_recherche_developpement}).

\subsection{gestion de projet}
La gestion de projet suit une méthode agile et s'appuie sur les outils de gestion de projet fourni par Github.

Chaque semaine, les tâches à réaliser sont définies avec mon encadrant.
Ces taches sont ajoutées sur github dans l'onglet Issues et leurs avancements est visible sur un tableau kanban.

Lorsque je termine une tache, j'ouvre un "pull request" pour que mon encadrant valide mon travail.

\subsection{déploiement continu}
% todo : parler de circleci
