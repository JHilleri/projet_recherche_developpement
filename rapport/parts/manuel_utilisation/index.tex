\chapter{Manuel d'utilisation}

\section{recherche locale}
La recherche locale s'exécute en invite de commande et nécessite les paramètres suivants :
\begin{itemize}
    \item Chemin vers le fichier d'instance.
    \item Taille maximum des lots.
    \item Chemin vers le fichier de solution.
    \item Durée minimum de la recherche locale.
\end{itemize}
Si le fichier de la solution existe déjà, son contenu seras écrasé.

\section{bibliothèque}
Pour l'utiliser dans un projet, il faut :
- ajouter le répertoire include de la bibliothèque.
% // todo : ajouter une capture d'écran
- ajouter le fichier lib
% // todo :  ajouter une capture d'écran
\section{visionneuse de solution}

Pour visionner une solution, il faut ouvrir le fichier view.html, celui-ci doit être dans le même fichier que la solution.
Le nom du ficher de solution peut être donné à la page web avec le paramètre filename.
Par exemple, si le fichier de la solution est solution.json et se trouve à la racine de l'ordinateur, il faut entrer l'URL suivante dans le navigateur.


\detokenize{file:///C:/view.html?filename=res_first.json}



Pour visionner une solution, il faut un accès à internet pour charger aux bibliothèques utilisées.


