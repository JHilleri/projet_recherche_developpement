\chapter{Sous-problème initial}

\label{appendix:modelisation_initiale}

Dans ce sous-problème où les lots et les ordres des jobs sont fixés.

% Ce modèle ce base sur celui proposé par mon encadrant Hugo Chevroton \cite{second_papier}.

Il est possible de faire partir les lots au plus tôt pour minimiser les pénalités de retard ou les retarder pour réduire les couts d'inventaire.
Pour connaitre l'impact de ces décisions sans résoudre le problème de livraison pour chaque combinaison,
mon encadrant à trouvé une solution, construire une fonction $F_k$ qui donne pour un lot $k$ les pénalités de retard en fonction de la date de départ du lot.
Cette fonction n'est calculée qu'une seul fois pour chaque lot car les compositions des lots sont fixes.
Une fois cette fonction calculée, il est possible de résoudre la date de départ de chaque lot séparément.

\section{Paramètres}
\begin{labeling}{paramètres}
	\item [$m$]
	Nombre de machines
	\item [$n$]
	Nombre de jobs.
	\item [$V$]
	Nombre de lots.
	\item [$j_k$]
	Dernier job du lot $k$.
	\item [$J^k$] 
	Liste ordonnée des jobs du lot $k$.
	\item [$p_{i, j}$]
	Durée de travail sur la machine $i$ pour le job $j$
	\item [$a_k$]
	Date de départ au plus tôt du lot $k$.
	\item [$b_k$]
	Date de départ du lot k à partir de laquelle les couts d'inventaires sont minimisés.
	\item [$\lambda_k$]
	Nombre de segment de la fonction $F_k$
	\item [$\alpha_{k, s}$]
	Couts de chaque unité de temps dans le segment $s$ de la fonction $F_k$.
	\item [$c_{k, s}$]
	Couts des pénalités de retard pour le $s$\ieme segment de la fonction $F_k$.
	\item [$t_{k, s}$]
	Date de début du $s$\ieme segment de la fonction $F_k$, $t_{k, \lambda_k+1}$ vaut $b_k$.
	\item [$h_{j}^{WIP}$] 
	Cout d'inventaire pendant la production du job $j$.
	\item [$h_{j}^{FIN}$] 
	Cout d'inventaire après la production du job $j$.
\end{labeling}


\section{Variables}

\begin{labeling}{variables}
	\item [$C_{i, j}$]
	Date de fin du travail de la machine $i$ sur le job $j$.
	\item [$x_{k, s}$]
	Vaut 0 ou 1 si la date de départ du lot $k$ est dans le segment $s$ de la fonction $F_k$.
	\item [$d_{k, s}$]
	Représente le temps écoulé dans le segment $s$ avant le départ du lot $k$.
	\item [$IC^{WIP}$]
	Cout d'inventaire pendant la production.
	$$IC^{WIP}=\sum_{j=1}^{n} \sum_{i=1}^{m-1} \left(C_{i+1, j} - p_{i+1, j} - C_{i,j} \right) h_j^{WIP}$$
	\item [$IC^{FIN}$]
	Cout d'inventaire entre la fin de la production et la livraison des jobs.
	$$IC^{FIN} = \sum_{k=1}^{V} \sum _{j\in j^k} \left(C_{m, j_k} - p_{m, j_{n_k}^k} - C_{m, j} \right) h_j^{FIN}$$
	\item [$IC$]
	Couts d'inventaire total.
	$$IC = IC^{WIP} + IC^{FIN}$$
\end{labeling}


\section{Objectif}
Somme des coûts à minimiser
$$IC + \sum_{k=1}^{V} \sum_{s=1}^{\lambda_k} \left(x_{k, s} c_{k, s}+ d_{k,s} \alpha_{k, s} \right)$$

\section{Contraintes}

\begin{align}
	C_{j_k, m} \leq \sum_{s=1}^{\lambda_k} \left( x_{k,s} t_{k, s} + d_{k, s} \right) &  & \forall k \in \left\{i, \cdots, V \right\}                                                    \\
	\sum_{s=1}^{\lambda_k} x_{k, s} = 1                                               &  & \forall k \in \left\{i, \cdots, V \right\}                                                    \\
	d_{k, s} \leq x_{k, s} \left( t_{k, s+1 - t_{k, s} -1} \right)                    &  & \forall k \in \left\{i, \cdots, V \right\}, \forall s \in \left\{1, \cdots, \lambda_k\right\} \\
	0 \leq d_{k, \lambda_k}                                                           &  & \forall k \in \left\{i, \cdots, V \right\}
\end{align}
\begin{itemize}
	\item
	      Lien entre la fin du dernier job d'un lot et le départ du lot
	      \begin{align}
		      C_{j_k, m} \leq \sum_{s=1}^{\lambda_k} \left( x_{k,s} t_{k, s} + d_{k, s} \right) &  & \forall k \in \left\{i, \cdots, V \right\}
	      \end{align}
	\item
	      La date de départ en livraison ne peut faire partie que d'un segment de la fonction $F_k$.
	      \begin{align}
		      \sum_{s=1}^{\lambda_k} x_{k, s} = 1 &  & \forall k \in \left\{i, \cdots, V \right\}
	      \end{align}
	\item
	      Chaque segment n'est utilisé que s'il contient la date de départ du lot.
	      \begin{align}
		      d_{k, s} \leq x_{k, s} \left( t_{k, s+1 - t_{k, s} -1} \right) &  & \forall k \in \left\{i, \cdots, V \right\}, \forall s \in \left\{1, \cdots, \lambda_k\right\} \\
		      0 \leq d_{k, \lambda_k}                                        &  & \forall k \in \left\{i, \cdots, V \right\}
	      \end{align}
\end{itemize}

Les contraintes suivantes servent à accélérer la résolution du problème.
\begin{itemize}
	\item
	      La date de fin optimale du lot $k$ est entre les dates $a_k$ et $b_k$ calculé au préalable.
	      \begin{align}
		      a_k \leq C_{j_k, m} \leq b_k &  & \forall k \in \left\{1,\cdots,V\right\}
	      \end{align}
	\item
	      Le temps écoulé dans un segment de $F_k$ avant le départ d'un lot ne doit pas faire partir le lot après $b_k$.
	      \begin{align}
		      d_{k, \Lambda_k} \leq t_{k s+1} - t_{k, s} &  & s \in \left\{1,\cdots,\lambda_k\right\}, \forall k \in \left\{1,\cdots,V\right\}
	      \end{align}
	\item
	      La date du départ d'un lot ne peut pas précéder le début du segment auquel il appartient.
	      \begin{align}
		      x_{k, s^t} t_{k, s} \leq C_{j_k, m} &  & s \in \left\{1,\cdots,\lambda_k\right\}, \forall k \in \left\{1,\cdots,V\right\}
	      \end{align}
\end{itemize}
