\subsection{Paramètres}
\begin{labeling}{paramètres}
	\item [$m$]
	Nombre de machines
	\item [$n$]
	Nombre de jobs.
	\item [$V$]
	Nombre de lots.
	\item [$N_k$]
	Nombre de job du lot $k$.
	\item [$p_{i, j}$]
	Durée de travail sur la machine $i$ pour le job $j$.
	\item [$h_{j}^{WIP}$] 
	Cout d'inventaire pendant la production du job $j$.
	\item [$h_{j}^{FIN}$] 
	Cout d'inventaire après la production du job $j$.
	\item [$a_k$]
	Date de départ au plus tôt du lot $k$.
	\item [$b_k$]
	Date de départ du lot k a partir de laquelle les couts d'inventaires sont minimisé.
	\item [$\lambda_k$]
	Nombre de segment de la fonction $F_k$
	\item [$\alpha_{k, s}$]
	Cout de chaque unité de temps dans le segment $s$ de la fonction $F_k$.
	\item [$c_{k, s}$]
	Cout des pénalités de retard pour le $s$\ieme segment de la fonction $F_k$.
	\item [$t_{k, s}$]
	Date de début du $s$\ieme segment de la fonction $F_k$, $t_{k, \lambda_k+1}$ vaut $b_k$.
	
\end{labeling}


\subsection{Variables}

\begin{labeling}{variables}
	\item [$x_{k, s}$]
	Vaut 0 ou 1 si la date de départ du lot $k$ est dans le segment $s$ de la fonction $F_k$.
	\item [$d_{k, s}$]
	Représente le temps écoulé dans le segment $s$ avant le départ du lot $k$.
	\item [$C_{i, j}$]
	Date de fin du travail de la machine $i$ sur le job $j$.
	
	\item [$y_{j1,j2}$]
	Variable de précédences en production, vaut 1 si le job $j1$ est effectué avant le job $j2$.
	\item [$j_k$]
	Dernier job du lot $k$.
	\item [$F_k$] 
	Date de fin de production du lot $k$.
	
	\item [$IC^{WIP}$]
	Cout d'inventaire pendant la production.
	$$IC^{WIP}= \sum_{j=1}^{n} \sum_{i=1}^{m-1} \left(C_{i+1, j} - p_{i+1, j} - C_{i, j} \right) h_j^{WIP}$$
	\item [$IC^{FIN}$]
	Cout d'inventaire entre la fin de la production et la livraison des jobs.
	$$IC^{FIN} = \sum_{k=1}^{V} \sum _{j\in j^k} \left( F_k - C_{m, j} \right) h_j^{FIN}$$
	\item [$IC$]
	Couts d'inventaire total.
	$$IC = IC^{WIP} + IC^{FIN}$$
\end{labeling}


\subsection{Objectif}
Somme des coûts à minimiser
$$IC + \sum_{k=1}^{V} \sum_{s=1}^{\lambda_k} \left(x_{k, s} c_{k, s}+ d_{k,s} \alpha_{k, s} \right)$$

\subsection{Contraintes}

\begin{itemize}
	\item
	      Lien entre la fin du dernier job d'un lot et le départ du lot
	      \begin{align}
		      C_{j_k, m} \leq \sum_{s=1}^{\lambda_k} \left( x_{k,s} t_{k, s} + d_{k, s} \right) &  & \forall k \in \left\{i, \cdots, V \right\}
	      \end{align}
	\item
	      La date de départ en livraison ne peut faire partie que d'un segment de la fonction $F_k$.
	      \begin{align}
		      \sum_{s=1}^{\lambda_k} x_{k, s} = 1 &  & \forall k \in \left\{i, \cdots, V \right\}
	      \end{align}
	\item
	      Chaque segment n'est utilisé que si il contient la date de départ du lot.
	      \begin{align}
		      d_{k, s} \leq x_{k, s} \left( t_{k, s+1 - t_{k, s} -1} \right) &  & \forall k \in \left\{i, \cdots, V \right\}, \forall s \in \left\{1, \cdots, \lambda_k\right\} \\
		      0 \leq d_{k, \lambda_k}                                        &  & \forall k \in \left\{i, \cdots, V \right\}
		  \end{align}
	% 
	\item
	      Chaque job est soit le successeur soit le prédécesseur de chacun des autres jobs.
	      \begin{align}
		      y_{j1,j2}+y_{j2,j1}=1 &  &
		      \forall j1,j2\in\left\{1,\dotsc,n\right\}, j1<j2
	      \end{align}
	\item
	      Les jobs ne peuvent pas être leurs propres successeurs.
	      \begin{align}
		      y_{j,j}=0 &  &
		      \forall j\in \left\{1,\cdots, n\right\}
	      \end{align}
	\item
	      Contrainte de gamme, les machines ne peuvent pas travailler sur plusieurs jobs simultanément.
	      Cette contrainte assure également que les jobs sont produits dans l'ordre prescrit.
	      \begin{align}
		      C_{i,j2} \geq C_{i,j1}+p_{i,j2}-M y_{j1,j2} &  &
		      \forall i\in\left\{1,\dotsc,m\right\}, \forall\left(j1,j2\right)\in\left\{1,\cdots,n\right\}^2
	      \end{align}
	\item
	      Contrainte de précédence, une machine ne peut pas travailler sur un job tant que la machine précédente n’a pas terminé de travailler dessus.
	      Cette contrainte assure également que les tâches des jobs sont produit dans l'ordre prescrit.
	      \begin{align}
		      C_{i,j}\geq C_{i-1,j}+p_{i,j} &  &
		      \forall j\in\left\{1,\cdots,n\right\},\forall i\in\left\{2,\dotsc,m\right\}
		  \end{align}
	\item 
		Un job est terminé quand toutes ces tâches sont terminées.
		\begin{align}
			f_j \geq C_{m, j} & & \forall j \in \left\{0,\dotsc,n\right\}
		\end{align}
	\item
	      Un lot ne peut être livré que si tous ces jobs sont terminés.
	      \begin{align}
		      F_k \geq C_{m, j} &  & \forall k \in \left\{1,\dotsc, n \right\}, \forall j \in J^k
	      \end{align}
\end{itemize}

Les contraintes suivantes servent à accélérer la résolution du problème.
\begin{itemize}
	\item
	      La date de fin du lot $k$ optimal est entre les dates $a_k$ et $b_k$ calculé au préalable.
	      \begin{align}
		      a_k \leq C_{j_k, m} \leq b_k &  & \forall k \in \left\{1,\cdots,V\right\}
	      \end{align}
	\item
	      Le temps écouler dans un segment de $F_k$ avant le départ d'un lot ne doit pas faire partir le lot après $b_k$.
	      \begin{align}
		      d_{k, \Lambda_k} \leq t_{k s+1} - t_{k, s} &  & s \in \left\{1,\cdots,\lambda_k\right\}, \forall k \in \left\{1,\cdots,V\right\}
	      \end{align}
	\item
	      La date du départ d'un lot ne peut pas précéder le début du segment auquel il appartient.
	      \begin{align}
		      x_{k, s^t} t_{k, s} \leq C_{j_k, m} &  & s \in \left\{1,\cdots,\lambda_k\right\}, \forall k \in \left\{1,\cdots,V\right\}
	      \end{align}
\end{itemize}
