\documentclass{article}
\usepackage[utf8]{inputenc}

\title{The Integrated Production and Transportation
Scheduling Problem for a Product
with a Short Lifespan}

\author{H. Neil Geismar, Gilbert Laporte, Lei Lei, Chelliah Sriskandarajah}

\begin{document}
\maketitle

\textbf{Article}

\section*{Interest}


\section*{Keywords }
supply chain; production and distribution; logistics; genetic algorithms; memetic algorithms

\section*{Resume}

We will enumerate the main difference between the problem present in this paper and the problem we try to solve. The vehicle is unique and with capacity, all the product has the same lifespan. This two aspect give to the authors of this paper to define easily the set of possible trip. Moreover, the objective function is minimize the maxspan (the date when the last trip come back to the depot).

\vspace{0.4cm}



The authors present lower bound and propose a genetic and memetic algorithm to solve this problem. The structure of the solution is composed of a permutation of client with the classic selection, cross over and mutation operator. All the complexity of the algorithm take place in the fitness function which develop meta heuristic to determine the optimal trips with respect to the initial permutation \textit{O($n^2$)}.


\section*{Interesting papers}
\begin{itemize}
\item
\end{itemize}

\end{document}